\documentclass[12pt]{article}
\usepackage{float}
\usepackage[ruled,vlined,linesnumbered,algo2e]{algorithm2e}
\usepackage{amsmath,amssymb}
\usepackage{makecell}
\usepackage{tikz}
\usepackage[a4paper, total={6.5in, 9in}]{geometry}

\newcommand*\circled[1]{\tikz[baseline=(char.base)]{
   \node[shape=circle,draw=red,inner sep=1pt] (char) {#1};}}
\setlength\parindent{0pt} %% Do not touch this
\DeclareMathOperator{\phiAb}{\phi_{A,\mathbf{b}}}
\newtheorem{lemma}{Lemma}
\newtheorem{theorem}{Theorem}
\newtheorem{definition}{Definition}
%% -----------------------------
%% TITLE
%% -----------------------------
\title{A short review of beamforming techniques for passive array processing} %% Assignment Title
\author{Wissal Ghamour, Julien Gleyze and Nathan Rousselot}
%% Change "\today" by another date manually
%% -----------------------------
%% -----------------------------

%% %%%%%%%%%%%%%%%%%%%%%%%%%
\begin{document}
%\setlength{\droptitle}{-5em}    
%% %%%%%%%%%%%%%%%%%%%%%%%%%
\maketitle
% --------------------------
% Start here
% --------------------------
\section{Introduction}
In this document, we will cover a range of techniques specific to array processing. Array processing is a branch of signal processing dedicated to the processing of signals produced or received by an array of elements. Those element can be antennas in passive setups, or transducers in the context of radar or sonar processing. In the following, we will focus on the passive setup, meaning we work with an array of sensors. This class of problem has a wide range of applications, from radio astronomy to wireless communications. All those applications share common challenges, such as the estimation of the direction of arrival (DOA) of a signal, or optimizing the Signal-to-Interference-plus-Noise ratio (SINR). In this introductory document, we will focus ourselves on the impact of the choice of beamforming techniques, ranging from Conventional Beamforming to more robust and adaptative methods.

\section{Beamforming Techniques}

% %%%%%%%%%%%%%%%%%%%


\end{document}